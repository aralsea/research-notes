\documentclass[uplatex,a4paper,11pt]{jsarticle}


% 数式
\usepackage{amsthm}
\usepackage{amsmath,amsfonts,amssymb}
\usepackage{color}
\usepackage{comment}
\usepackage{ulem}
\usepackage[all]{xy}
\def\objectstyle{\displaystyle}




\newtheoremstyle{mystyle}% % スタイル名
 {}% % 上部スペース
 {}% % 下部スペース
 {\normalfont}% % 本文フォント
 {}% % インデント量
 {\bf}% % 見出しフォント
 {}% % 見出し後の句読点, '.'
 { }% % 見出し後のスペース, ' ' or \newline
 {\underline{\thmname{#1}\thmnumber{#2}\thmnote{(#3)}}}%
 % 見出しの書式 (can be left empty, meaning `normal')
\theoremstyle{mystyle} % スタイルの適用

\newtheorem{theorem}{Thm.}[section]
\newtheorem{proposition}{Prop.}[section]
\newtheorem{lemma}{Lem.}[section]
\newtheorem{corollary}{Cor.}[section]

\makeatletter % use at mark
\renewenvironment{proof}[1][\proofname]{\par
 \pushQED{\qed}%
 \normalfont \topsep6\p@\@plus6\p@\relax
 \trivlist
 \item[\hskip\labelsep
 \itshape
 {\bf\underline{#1}}]\ignorespaces
 % {\bf\underline{#1}\@addpunct{.}}]\ignorespaces % ピリオドあり
}{%
 \popQED\endtrivlist\@endpefalse
}
\makeatother % end at mark

\renewcommand{\abstractname}{}
\renewcommand{\theenumi}{(\arabic{enumi})}
\renewcommand{\labelenumi}{(\arabic{enumi})}

\newcommand{\emphasis}[1]{\textgt{\textcolor{magenta}{#1}}}
\DeclareMathOperator{\Aut}{Aut}
\DeclareMathOperator{\Spec}{Spec}
\DeclareMathOperator{\Auteq}{Auteq}
\DeclareMathOperator{\Hom}{Hom}
\DeclareMathOperator{\id}{id}


\begin{document}

\title{}
\author{}
\date{}
%\maketitle
特にことわらない限り,$X$を小平I\hspace{ - .1em}I\hspace{ - .1em}I型曲線($=$2つの$\mathbb{P}^1$が1点で接している特異曲線)とする.
\begin{lemma}\label{auto_of_3}
	$p,q \in X$を異なる既約成分上にある非特異閉点とする.$p,q$を動かさない$f \in \Aut{X}$は$k^\times$によりパラメトライズされる.
\end{lemma}
\begin{proof}
	$X$を$\{y=x^2\}\cup\{y=0\} \subset \mathbb{A}^2$の射影化$\{yz=x^2\}\cup\{y=0\} \subset \{[x:y:z]\}=\mathbb{P}^2$とみる.これにより$p=[0:1:0] \in \{yz=x^2\}, q=[1:0:0] \in \{y=0\}$だとする.
	$s \in X$を$X$の(唯一の)特異点$[0:0:1]$とし,$n \colon L_1 \sqcup L_2 \to X$を$X$の正規化とする.$ L_1,L_2 \cong \mathbb{P}^1$の座標を$$n|_{L_1}(0)=s, n|_{L_1}(\infty)=x, n|_{L_2}(0)=s, n|_{L_2}(\infty)=y,$$となるようにとる.このとき$f$が誘導する$L_1 \sqcup L_2$の自己同型$\tilde{f}$は,上の座標により$L_1,L_2$上で$0$でない定数倍写像により与えられる.($f$は$X$の既約成分を保つので各$L_i$の自己同型を誘導し,座標の取り方より$0$と$\infty$を固定する$\mathbb{P}^1$の自己同型になるから.)$\tilde{f}|_{L_1},\tilde{f}|_{L_2}$をそれぞれ$a$倍,$b$倍だとする.

	以上の座標の設定により,$s$のアファイン近傍$U=X - \{p,q\}$に$n$を制限したものは単射な$k$代数準同型
	$$k[x, y]/y(y - x^2) \to k[x_1]\times k[x_2]$$
	$$x \mapsto (x_1, x_2)$$
	$$y \mapsto (x_1^2, 0)$$
	の$\Spec$をとったものになる($k[x_1],k[x_2]$がそれぞれ$L_1, L_2$に対応).$f$が$p,q$を保つという条件より,$f$は$U$の自己同型,すなわち$k[x, y]/y(y - x^2)$の自己同型$\varphi$を誘導する.また$\tilde{f}|_{L_1},\tilde{f}|_{L_2}$が$\mathbb{P}^1$の$a$倍,$b$倍写像だったので,結局以下の可換図式が得られる.

	\[
		\xymatrix{
			k[x_1]\times k[x_2] \ar[r]^{(x_1 \mapsto ax_1, x_2 \mapsto bx_2)} & k[x_1]\times k[x_2]\\
			k[x, y]/y(y - x^2) \ar[u] \ar[r]^\varphi & k[x, y]/y(y - x^2) \ar[u]
		}
	\]
	このとき$a=b$で$\varphi(x)=ax, \varphi(y)=a^2y$であることを示す.まず$y \in k[x, y]/y(y - x^2)$について考えると,
	\[
		\xymatrix{
			(x_1^2,0) \ar@{|->}[r] & (a^2x_1^2, 0)\\
			y \ar@{|->}[u] \ar@{|->}[r] & \varphi(y) \ar@{|->}[u]
		}
	\]
	となり,縦の射は単射だったから$\varphi(y)=a^2y$となる.次に$\varphi(x) \in k[x,y]/y(y - x^2)$の$k[x,y]$へのリフトの1つを$g(x,y)$とすると,
	\[
		\xymatrix{
			(x_1,x_2) \ar@{|->}[r] & (ax_1, bx_2)\\
			x \ar@{|->}[u] \ar@{|->}[r] & g(x,y) \mod y(y - x^2) \ar@{|->}[u]
		}
	\]
	より,$$g(x_1,x_1^2)=ax_1, g(x_2, 0) = bx_2$$が成り立つ.2つ目の式より多項式$h(x,y)$を使って$$g(x.y)=bx + yh(x,y)$$とかけ,1つ目の式より$$bx_1 + x_1^2h(x_1,x_1^2)=ax_1$$となる.係数比較により$a=b$となり,\[
		\xymatrix{
			(x_1,x_2) \ar@{|->}[r] & (ax_1, ax_2)\\
			x \ar@{|->}[u] \ar@{|->}[r] & \varphi(x) \ar@{|->}[u]
		}
	\]
	となるから,縦の射の単射性より$\varphi(x)=ax$である.
\end{proof}

\begin{lemma}\label{iso_of_line_bundles}
	$p,q \in X$を異なる既約成分上にある非特異閉点とし,$p$と$q$を入れ替えるような$X$の自己同型で,involutionであるものの1つを$\sigma$とする.(このような$\sigma$は2つある.)$r \in X$を非特異閉点とし,$s = \sigma(r)\in X$と定めると,$$\mathcal{O}(p + q)\cong \mathcal{O}(r + s)$$である.
\end{lemma}
\begin{proof}
	$p,q$で$1$位の極をもち$r,s$で$1$位の零点を持つような$X$上の有理型関数を構成すればよい.Lem.\ref{auto_of_3} と同じ記法・座標で考える.$\sigma$としてあり得るのは(Lem.\ref{auto_of_3} の記法のもとで)$x_1 \mapsto x_2, x_2 \mapsto x_1$または$x_1 \mapsto - x_2, x_2 \mapsto - x_1$の2通りである.

	$r$は(Lem.\ref{auto_of_3}の座標で)$[1:1:1]\in\{yz=x^2\}$だとしてよい.また$s$の座標が$[a:0:1]$($a \in k^\times$)だとする.このとき$p, r$は$n$により$L_1 \cong \mathbb{P}^1$上の点$\infty, 1$に対応し,$q,s$は$n$により$L_2 \cong \mathbb{P}^1$上の点$\infty, a$に対応する.

	そこで,$L_1$上の$\infty$に$1$位の極,$1$に$1$位の零点をもつ有理型関数$f$と,$L_2$上の$\infty$に$1$位の極,$a$に$1$位の零点をもつ有理型関数$g$をうまく取り,それらを貼り合わせて求める$X$上の有理型関数が作れることを示せばよいが,$f(x_1)=x_1 - 1, g(x_2)=x2 - 1$とおけばこれらは$\Spec k[x,y]/y(y - x^2)$上の正則関数$h(x,y)=x - 1$の引き戻しだから,$X$上で貼り合う.
\end{proof}

\begin{lemma}\label{calc_of_twist_functors}
	(I型でも正しい.)

	$x \in X$を非特異閉点とする.
	$D^b(X)$のspherical object $\mathcal{O}_x, \mathcal{O}$に伴うtwist functorをそれぞれ$T_x, T_{\mathcal{O}}$とすると,$D^b(X)$において次が成り立つ.
	\begin{enumerate}
		\item $T_x T_\mathcal{O}T_x \cong T_\mathcal{O}T_x T_\mathcal{O}$ (Braid relation)
		\item $T_x\cong\mathcal{O}(x) \otimes - $
		\item $T_{\mathcal{O}}(\mathcal{O}_x)\cong\mathcal{O}( - x)[1]$
		\item $T_{\mathcal{O}}(\mathcal{O}(x))\cong\mathcal{O}_x$
		\item $T_{\mathcal{O}}(\mathcal{O})\cong\mathcal{O}$
	\end{enumerate}
\end{lemma}
\begin{proof}
	\begin{enumerate}
		\item $\sum_{i} \dim \Hom^i (\mathcal{O}, \mathcal{O}_x)=1$だから,組$(\mathcal{O}, \mathcal{O}_x)$は\cite{ST01}の意味での$A_2$ - configurationである.よって\cite{ST01} Thm.1.2により正しい.
	\end{enumerate}
\end{proof}

\begin{lemma}\label{calc_of_twist_functors2}
	(I型でも正しい.)

	$x, y \in X$を非特異閉点とする(同じ既約成分上にあってもok).このとき
	\begin{enumerate}
		\item $T_\mathcal{O}(\mathcal{O}(x + y))\cong \mathcal{O}( - x - y)[1]$
		\item $T_\mathcal{O}(\mathcal{O}(x - y))\cong \mathcal{O}(x - y)$
	\end{enumerate}
	が成り立つ.
\end{lemma}
\begin{proof}
	\begin{enumerate}
		\item Braid relation $T_x T_\mathcal{O}T_x \cong T_\mathcal{O}T_x T_\mathcal{O}$ に$\mathcal{O}(y)$を代入すると$$T_x T_\mathcal{O}T_x(\mathcal{O}(y))\cong \mathcal{O}(x)\otimes T_\mathcal{O}(\mathcal{O}(x + y))$$$$T_\mathcal{O}T_x T_\mathcal{O}(\mathcal{O}(y))\cong T_\mathcal{O}T_x (\mathcal{O}_y) \cong T_\mathcal{O}(\mathcal{O}_y) \cong \mathcal{O}( - y)[1]$$となるからよい.
		\item Braid relation $T_x T_\mathcal{O}T_x \cong T_\mathcal{O}T_x T_\mathcal{O}$ に$\mathcal{O}_y$を代入すると$$T_x T_\mathcal{O}T_x(\mathcal{O}_y)\cong T_x T_\mathcal{O}(\mathcal{O}_y) \cong T_x(\mathcal{O}( - y)[1])\cong \mathcal{O}(x - y)[1]$$$$T_\mathcal{O}T_x T_\mathcal{O}(\mathcal{O}_y)\cong T_\mathcal{O}T_x (\mathcal{O}( - y)[1]) \cong T_\mathcal{O}(\mathcal{O}(x - y)[1])$$となるからよい.
	\end{enumerate}
\end{proof}

\begin{lemma}\label{criterion_to_be_id}
	(I型でも正しい.)

	$x, y \in X$を,異なる既約成分上にある非特異閉点とする.$F \in \Auteq{D^b(X)}$が
	\begin{itemize}
		\item $F(\mathcal{O}) \cong \mathcal{O}$
		\item $F(\mathcal{O}_x) \cong \mathcal{O}_x$
		\item $F(\mathcal{O}_y) \cong \mathcal{O}_y$
	\end{itemize}
	を満たすならば,$f \in \Aut{X}$により$F\cong f^*$となる.さらにこの$f$は$x,y$を保つ.
\end{lemma}
\begin{proof}
	(I\hspace{ - .1em}I\hspace{ - .1em}I型のとき)$L=\mathcal{O}(x + y)$とおくと,これはampleである.($X=\{Y=0\}\cup\{YZ=X^2\} \subset \mathbb{P}^2$, $x=[1:0:0], y=[0:1:0]$とみたときに,
	$\mathbb{P}^2$のhyperplane $Z=0$による$X$のhyperplane sectionは$x + 2y$なので,$\mathcal{O}(x + 2y), \mathcal{O}(2x + y)$がvery ampleになり,従って$L^3=\mathcal{O}(3x + 3y)$もvery ampleだから.)

	あとは\cite{Sib14} Lem 3.3と同じ.条件から$F(L^{\otimes m})\cong L^{\otimes m}$が任意の整数$m$について成り立つ.これらの同型はそのままでは自然な同型ではないかもしれないが,これが$\oplus_{m=0}^\infty H^0(L^{\otimes m})$に誘導する代数の同型を考え,そこから$X$の自己同型$f$を作ってやるとBondal - Orlov reconstructionと同じ議論で$F \cong f^*$となる.($D^b(X)$の ample sequence $\{L^{\otimes m}\}_{m \in \mathbb{Z}}$上で同型→全体に伸ばす.)
\end{proof}
\begin{lemma}\label{calc_around_z}
	$x, y, z \in X$を非特異閉点とし,$F=(T_xT_\mathcal{O}T_y)^2$とする.このとき$$F(\mathcal{O}(z))\cong \mathcal{O}(2x - z)[1]$$である.
\end{lemma}
\begin{proof}
	\begin{align}
		F(\mathcal{O}(z - y))
		 & = T_xT_\mathcal{O}T_yT_xT_\mathcal{O}T_y(\mathcal{O}(z))      \\
		 & \cong T_xT_\mathcal{O}T_yT_xT_\mathcal{O}(\mathcal{O}(y + z)) \\
		 & \cong T_xT_\mathcal{O}T_yT_x(\mathcal{O}( - y - z)[1])        \\
		 & \cong T_xT_\mathcal{O}(\mathcal{O}(x - z)[1])]                \\
		 & \cong T_x(\mathcal{O}(x - z)[1])                              \\
		 & \cong \mathcal{O}(2x - z)[1]
	\end{align}
\end{proof}
\begin{theorem}\label{G_2 relation for the III curve}
	$x, y \in X$を非特異閉点とする.$\Auteq D^b(X)$において等式$$(T_xT_\mathcal{O}T_y)^4=[2]$$が成り立つ.
\end{theorem}
\begin{proof}
	$F=(T_xT_\mathcal{O}T_y)^2$とおく.Lem.\ref{iso_of_line_bundles}のような$x$と$y$を入れ替えるinvolutionの1つを$\sigma$とする.まず$$(\mathcal{O}(x - y)\otimes - )\circ\sigma^* F[ - 1]\cong\id$$であることを示す.
	\begin{align}
		(\mathcal{O}(x - y)\otimes - )\circ\sigma^* F[ - 1](\mathcal{O})
		 & = (\mathcal{O}(x - y)\otimes - )\circ\sigma^*T_xT_\mathcal{O}T_yT_xT_\mathcal{O}T_y(\mathcal{O})[ - 1]     \\
		 & \cong (\mathcal{O}(x - y)\otimes - )\circ\sigma^*T_xT_\mathcal{O}T_yT_xT_\mathcal{O}(\mathcal{O}(y))[ - 1] \\
		 & \cong (\mathcal{O}(x - y)\otimes - )\circ\sigma^*T_xT_\mathcal{O}T_yT_x(\mathcal{O}_y)[ - 1]               \\
		 & \cong (\mathcal{O}(x - y)\otimes - )\circ\sigma^*T_xT_\mathcal{O}(\mathcal{O}_y)[ - 1]                     \\
		 & \cong (\mathcal{O}(x - y)\otimes - )\circ\sigma^*T_x(\mathcal{O}( - y))                                    \\
		 & \cong (\mathcal{O}(x - y)\otimes - )\circ\sigma^*(\mathcal{O}(x - y))                                      \\
		 & \cong (\mathcal{O}(x - y)\otimes(\mathcal{O}(y - x))                                                       \\
		 & \cong \mathcal{O}
	\end{align}
	\begin{align}
		(\mathcal{O}(x - y)\otimes - )\circ\sigma^* F[ - 1](\mathcal{O}_x)
		 & = (\mathcal{O}(x - y)\otimes - )\circ\sigma^*T_xT_\mathcal{O}T_yT_xT_\mathcal{O}T_y(\mathcal{O}_x)[ - 1]  \\
		 & \cong (\mathcal{O}(x - y)\otimes - )\circ\sigma^*T_xT_\mathcal{O}T_yT_xT_\mathcal{O}(\mathcal{O}_x)[ - 1] \\
		 & \cong (\mathcal{O}(x - y)\otimes - )\circ\sigma^*T_xT_\mathcal{O}T_yT_x(\mathcal{O}( - x))                \\
		 & \cong (\mathcal{O}(x - y)\otimes - )\circ\sigma^*T_xT_\mathcal{O}(\mathcal{O}(y))                         \\
		 & \cong (\mathcal{O}(x - y)\otimes - )\circ\sigma^*T_x(\mathcal{O}_y)                                       \\
		 & \cong (\mathcal{O}(x - y)\otimes - )\circ\sigma^*(\mathcal{O}_y)                                          \\
		 & \cong \mathcal{O}(x - y)\otimes(\mathcal{O}_x)                                                            \\
		 & \cong \mathcal{O}_x                                                                                       \\
	\end{align}

	\begin{align}
		(\mathcal{O}(x - y)\otimes - )\circ\sigma^* F[ - 1](\mathcal{O}_y)
		 & = (\mathcal{O}(x - y)\otimes - )\circ\sigma^*T_xT_\mathcal{O}T_yT_xT_\mathcal{O}T_y(\mathcal{O}_y)[ - 1]  \\
		 & \cong (\mathcal{O}(x - y)\otimes - )\circ\sigma^*T_xT_\mathcal{O}T_yT_xT_\mathcal{O}(\mathcal{O}_y)[ - 1] \\
		 & \cong (\mathcal{O}(x - y)\otimes - )\circ\sigma^*T_xT_\mathcal{O}T_yT_x(\mathcal{O}( - y))                \\
		 & \cong (\mathcal{O}(x - y)\otimes - )\circ\sigma^*T_xT_\mathcal{O}(\mathcal{O}(x))                         \\
		 & \cong (\mathcal{O}(x - y)\otimes - )\circ\sigma^*T_x(\mathcal{O}_x)                                       \\
		 & \cong (\mathcal{O}(x - y)\otimes - )\circ\sigma^*(\mathcal{O}_x)                                          \\
		 & \cong \mathcal{O}(x - y)\otimes(\mathcal{O}_y)                                                            \\
		 & \cong \mathcal{O}_y                                                                                       \\
	\end{align}
	だから,Lem.\ref{criterion_to_be_id}よりある$x,y$を動かさない$X$の自己同型$f$があり$$(\mathcal{O}(x - y)\otimes - )\circ\sigma^* F[ - 1] \cong f^*$$となる.さらに$z \in X$を$x,y$と異なる非特異閉点としたとき,Lem.\ref{calc_around_z}により
	\begin{align}
		(\mathcal{O}(x - y)\otimes - )\circ\sigma^* F[ - 1](\mathcal{O}(z))
		 & \cong (\mathcal{O}(x - y)\otimes - )\circ\sigma^* \mathcal{O}(2x - z) \\
		 & \cong \mathcal{O}(x - y)\otimes \mathcal{O}(2y - \sigma(z))           \\
		 & \cong \mathcal{O}(x + y - \sigma(z))                                  \\
	\end{align}
	となる($\sigma^{ - 1}=\sigma$に注意する).これはLem.\ref{iso_of_line_bundles}により$\mathcal{O}(z)$と同型である.つまり$(\mathcal{O}(x - y)\otimes - )\circ\sigma^* F[ - 1] \cong f^*$は$\mathcal{O}(z)$を保つ.$f$は自己同型だから連接層の完全列を保つことに注意して完全列$$0 \to \mathcal{O} \to \mathcal{O}(z)\to\mathcal{O}_z\to 0$$に$f^*$を施すと,$$0 \to \mathcal{O} \to \mathcal{O}(z)\to f^*\mathcal{O}_z\to 0$$という完全列を得る.単射$\mathcal{O} \to \mathcal{O}(z)$は(リーマンロッホより$\dim H^0(\mathcal{O}(z))=1$であることから)$k^\times$倍を除いて一意だから,結局この完全列より$f^*\mathcal{O}_z \cong \mathcal{O}_z$がわかる.つまり$f$は$z$を保ち,Lem.\ref{auto_of_3}より$f = \id$となる.

	以上より$$F \cong \sigma^*\circ (\mathcal{O}(y - x)\otimes - )[1]$$となる.これと$$\sigma^*\circ (\mathcal{O}(y - x)\otimes - ) \cong (\mathcal{O}(x - y)\otimes - )\circ \sigma^*$$により,$$F^2 \cong [2]$$である.
\end{proof}

\subsection*{Sibillaの矛盾?→解決した.}
計算してたらおかしな結果が出たので,どこか間違ってるはず.
\begin{proposition}[\cite{Sib14}Thm.3.6の証明の$n=2$の議論]\label{G_2 relation for the I_2 curve}
	$X$をI$_2$型曲線とし,$x, y \in X$を異なる既約成分上にある非特異閉点とする.また$x$と$y$を入れ替え特異点を保つような唯一のinvolutionを$\sigma \colon X \to X$とし,$F=(T_xT_\mathcal{O}T_y)^2$とする.このとき$x,y$を保つような自己同型$f \colon X \to X$があり,$$F \cong \textcolor{magenta}{(\mathcal{O}(x - y)\otimes - )\circ} f^*\sigma^*[1]$$となる.さらに$f$は2つの特異点を入れ替えるような唯一のinvolutionまたは$\id$で,$\sigma$と可換である.特に$$F^2 \cong[2]$$である.
\end{proposition}
\begin{proof}
	Lem.\ref{calc_of_twist_functors}を使って$$F(\mathcal{O})\cong \hbox{\sout{$\mathcal{O}[1]$}} \hspace{5pt}\textcolor{magenta}{\mathcal{O}(x - y)[1]}$$$$F(\mathcal{O}_x)\cong \mathcal{O}_y[1]$$$$F(\mathcal{O}_y)\cong \mathcal{O}_x[1]$$を示す.$\mathcal{O}_x$については
	\begin{align}
		T_xT_\mathcal{O}T_yT_xT_\mathcal{O}T_y(\mathcal{O}_x)
		 & \cong T_xT_\mathcal{O}T_yT_xT_\mathcal{O}(\mathcal{O}_x) \\
		 & \cong T_xT_\mathcal{O}T_yT_x(\mathcal{O}( - x)[1])       \\
		 & \cong T_xT_\mathcal{O}T_y(\mathcal{O}[1])                \\
		 & \cong T_xT_\mathcal{O}(\mathcal{O}(y)[1])                \\
		 & \cong T_x\mathcal{O}_y[1]                                \\
		 & \cong \mathcal{O}_y[1]                                   \\
	\end{align}
	となり,他も同様にわかる.よって関手$\textcolor{magenta}{(\mathcal{O}(x - y)\otimes - )}\circ\sigma^* \circ F[ - 1]$はLem.\ref{criterion_to_be_id}の条件を満たす.よってLem.\ref{criterion_to_be_id}より($\sigma^2=\id$に注意すると)$x,y$を保つような自己同型$f \colon X \to X$があり,$$F \cong \textcolor{magenta}{(\mathcal{O}(x - y)\otimes - )}\circ f^*\sigma^*[1]$$となる.
	さらに$f$が$X$の2つの特異点を入れ替えるか,保つかによってinvolutionまたは$\id$となることが(normalizationに誘導する自己同型を見ることで)わかる.
\end{proof}
\begin{proposition}
	Prop.\ref{G_2 relation for I_2 curve}の状況で,$z \in X$を非特異閉点とする.このとき
	$$F(\mathcal{O}(z - 2y))\cong \mathcal{O}( - z)[1]$$であり,$$\textcolor{magenta}{(\mathcal{O}(x - y)\otimes - )}\circ f^*\sigma^*(\mathcal{O}(z - 2y))[1]\cong \hbox{\sout{$\mathcal{O}(f(\sigma(z)) - 2x)[1]$}}\hspace{5pt}\textcolor{magenta}{\mathcal{O}(f(\sigma(z)) - x - y)[1]}$$である.
\end{proposition}
\begin{proof}
	Lem.\ref{calc_of_twist_functors},
	Lem.\ref{calc_of_twist_functors2}を使って計算すると
	\begin{align}
		F(\mathcal{O}(z - 2y))
		 & = T_xT_\mathcal{O}T_yT_xT_\mathcal{O}T_y(\mathcal{O}(z - 2y)) \\
		 & \cong T_xT_\mathcal{O}T_yT_xT_\mathcal{O}(\mathcal{O}(z - y)) \\
		 & \cong T_xT_\mathcal{O}T_yT_x(\mathcal{O}(z - y))              \\
		 & \cong T_xT_\mathcal{O}T_y(\mathcal{O}(x + z - y))             \\
		 & \cong T_xT_\mathcal{O}(\mathcal{O}(x + z))                    \\
		 & \cong T_x(\mathcal{O}( - x - z)[1])                           \\
		 & \cong \mathcal{O}( - z)[1]                                    \\
	\end{align}
	となる.
\end{proof}
\sout{$\mathcal{O}( - z)$と$\mathcal{O}(f(\sigma(z)) - 2x)$ではmulti - degreeが合わないので何かおかしい.}\textcolor{magenta}{あってる.}($\sigma$は既約成分を入れ替え,$f$は保つので$f(\sigma(z))$は$z$と異なる既約成分の上にある.)
\begin{comment}

\end{comment}
\begin{thebibliography}{99}
	\bibitem[Sib14]{Sib14} N. Sibilla, \textit{A note on mapping class group actions on derived categories}, Proceedings of the American Mathematical Society, 142(6):1837–1848, 2014.
	\bibitem[ST01]{ST01} P. Seidel and R. Thomas, \textit{Braid group actions on derived categories of coherent sheaves}, Duke Math. J., 108(1):37–108, 2001, MR 1831820, Zbl 1092.14025.
\end{thebibliography}
\end{document}
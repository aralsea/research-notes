\documentclass[uplatex,a4paper,11pt,dvipdfmx]{jsarticle}


% 数式
\usepackage{amsthm}
\usepackage{amsmath,amsfonts,amssymb}
\usepackage{color}
\usepackage{comment}
\usepackage{ulem}
\usepackage[all]{xy}
\usepackage{tikz-cd}
\def\objectstyle{\displaystyle}




\newtheoremstyle{mystyle}% % スタイル名
 {}% % 上部スペース
 {}% % 下部スペース
 {\normalfont}% % 本文フォント
 {}% % インデント量
 {\bf}% % 見出しフォント
 {}% % 見出し後の句読点, '.'
 { }% % 見出し後のスペース, ' ' or \newline
 {\underline{\thmname{#1}\thmnumber{#2}\thmnote{(#3)}}}%
 % 見出しの書式 (can be left empty, meaning `normal')
\theoremstyle{mystyle} % スタイルの適用

\newtheorem{theorem}{Thm.}[section]
\newtheorem{proposition}{Prop.}[section]
\newtheorem{lemma}{Lem.}[section]
\newtheorem{corollary}{Cor.}[section]

\makeatletter % use at mark
\renewenvironment{proof}[1][\proofname]{\par
 \pushQED{\qed}%
 \normalfont \topsep6\p@\@plus6\p@\relax
 \trivlist
 \item[\hskip\labelsep
 \itshape
 {\bf\underline{#1}}]\ignorespaces
 % {\bf\underline{#1}\@addpunct{.}}]\ignorespaces % ピリオドあり
}{%
 \popQED\endtrivlist\@endpefalse
}
\makeatother % end at mark

\renewcommand{\abstractname}{}
\renewcommand{\theenumi}{(\arabic{enumi})}
\renewcommand{\labelenumi}{(\arabic{enumi})}

\newcommand{\emphasis}[1]{\textgt{\textcolor{magenta}{#1}}}
\DeclareMathOperator{\Aut}{Aut}
\DeclareMathOperator{\Spec}{Spec}
\DeclareMathOperator{\Auteq}{Auteq}\DeclareMathOperator{\Coh}{Coh}
\DeclareMathOperator{\Hom}{Hom}
\DeclareMathOperator{\Ext}{Ext}

\DeclareMathOperator{\id}{id}
\DeclareMathOperator{\codim}{codim}
\DeclareMathOperator{\Cone}{Cone}
\newcommand{\Extsheaf}{\mathop{\mathcal{E}\! \mathit{xt}}\nolimits}
\newcommand{\Torsheaf}{\mathop{\mathcal{T}\! \mathit{or}}\nolimits}



\begin{document}

\title{}
\author{}
\date{}
%\maketitle
elliptic surface $\pi \colon S \to C$の定義などは\cite{Ueh15}に従う.
\begin{theorem}
	$\pi \colon S \to C$をelliptic surfaceとし,$G \subset S$を$(-2)$-curve,$a \in \mathbb{Z}$を整数とする.このとき$S$のspherical object $\mathcal{O}_G(a)$に付随するtwist functorの核$$P = \Cone(\mathcal{O}_G(a) \boxtimes \mathcal{O}_G(a)^\vee \xrightarrow{ev} \mathcal{O}_\Delta)$$は$S\times_C S$上の層である.
\end{theorem}
\begin{lemma}\label{flatness}
	$\pi \colon S \to C$をelliptic surfaceとする.この時$\pi$はflatである.
\end{lemma}
\begin{proof}
	$\pi(x)=y$とおくと,局所環の射$\mathcal{O}_{C, y} \to \mathcal{O}_{S, x}$が誘導される.$\mathcal{O}_{C, y} $はPIDで,$\mathcal{O}_{S, x}$は正則局所環だから特に整域である.よって$\mathcal{O}_{S, x}$はPID上のtorsion-free加群だからflatである.
\end{proof}
\begin{lemma}\label{Cartesian}
	図式
	\[
		\xymatrix{
			S \times_C S \ar[r] \ar[d]& S \times S \ar[d]^{\pi \times \pi}\\
			C \ar[r]^{\Delta_C} & C \times C
		}
	\]
	はカルテシアンである.
\end{lemma}
\begin{proof}
	いわゆるmagic diagram.
\end{proof}
\begin{lemma}\label{effective_divisor}
	$X = S \times_C S, Y = S \times S$とし,$i \colon X \to Y$をinclusionとする.このとき$Y$上のline bundle $L$ と完全列$$0 \to L \to \mathcal{O}_Y \to \mathcal{O}_X \to 0$$がある.ここで$\mathcal{O}_Y \to \mathcal{O}_X$は自然な全射である.
\end{lemma}
\begin{proof}
	$C$は非特異だから,$\Delta_C\colon C \to C \times C$により$C$は$C\times C$の中でlocal complete intersectionであり,完全列$$0 \to \mathcal{O}_{C\times C}(-\Delta) \to \mathcal{O}_{C \times C} \to \mathcal{O}_{\Delta} \to 0$$がある.さらにLem.\ref{flatness}より$\pi \times \pi$はflatだから,この完全列を$\pi \times \pi$でpullbackしてLem.\ref{Cartesian}の図式と組み合わせると求める完全列を得る.
\end{proof}
\begin{comment}
\begin{lemma}
	$Y$を非特異代数多様体とし,$X \subset Y$をregular immersionとする.また$C_{X/Y}$をそのconormal sheafとする.このとき任意の$F \in \Coh(X)$について$$Li^*(i_*F) \cong \bigoplus_{k = 0}^r (F\otimes \wedge^k C_{X/Y})[k]$$となる.
\end{lemma}
\begin{proof}
	$i_* \colon D^*(X) \to D^*(Y)$はexactだから$i_*Li^* \cong L(i_*i^*)$である.ここで$i_*i^* \cong -\otimes_{\mathcal{O}_Y}\mathcal{O}_X$だから,$L(i_*i^*) \cong -\otimes^L_{\mathcal{O}_Y}\mathcal{O}_X$となる.またregular immersionの仮定より,Koszul resolution
\end{proof}
\end{comment}
\begin{lemma}\label{counit_map}
	Lem.\ref{effective_divisor}の状況で$F \in \Coh(X)$とすると,随伴$Li^* \dashv i_*$に付随するcounit射$\varepsilon \colon Li^*(i_*F) \to F$が複体の$0$次コホモロジーに誘導する射は同型である.
\end{lemma}
\begin{proof}
	$i_*F \in \Coh(Y)$の bounded locally free resolution
	$$0 \to P^{-r}\to\cdots\to P^{-1}\to P^0 \to i_* F \to 0$$
	をとる.これにより得られる$D^b(Y)$での同型射を$\varphi \colon P^\bullet \to i_*F$とする.随伴の自然性より以下のような可換図式がある.
	\[
		\begin{tikzcd}
			\Hom_Y(i_*F, i_*F) \arrow[r, "\sim"]\arrow[d, "- \circ \varphi"] & \Hom_X(Li^*(i_*F), F) \arrow[d, "- \circ Li^*(\varphi)"]&\id \arrow[r, mapsto]\arrow[d, mapsto] & \varepsilon \arrow[d, mapsto]\\
			\Hom_Y(P^\bullet, i_*F) \arrow[r, "\sim"] & \Hom_X(Li^*(P^\bullet), F)&\varphi \arrow[r, mapsto] & \varepsilon \circ Li^*(\varphi)
		\end{tikzcd}
	\]
	$\varepsilon$の代わりに$\varepsilon \circ Li^*(\varphi)$が$0$次コホモロジー誘導する射について確かめればよい.上の可換図式より,$\varepsilon \circ Li^*(\varphi)$は$\varphi$を随伴でうつしたもの等しいが,下の段の随伴同型は derived でない随伴同型による複体の射の対応(から誘導されるもの)である.よって以下の$\varphi$
	\[
		\begin{tikzcd}
			\cdots\to &P^{-1}\arrow[d]&\to &P^0 \arrow[d, "\varphi^0"] &\to &0\arrow[d]\\
			&0&\to& i_*F & \to &0
		\end{tikzcd}
	\]
	を随伴でうつしたものは,$\varphi^0$を随伴でうつした$\psi^0$を用いて
	\[
		\begin{tikzcd}
			\cdots\to &i^*P^{-1}\arrow[d]&\to &i^*P^0 \arrow[d, "\psi^0"] &\to &0\arrow[d]\\
			&0&\to& F & \to &0
		\end{tikzcd}
	\]
	と表される射$\psi$である.derived でない順像逆像随伴の構成より,$\psi^0$は$\varphi^0$の$X$への制限に等しいから,$i^*$の右完全性より$0$次のコホモロジーに誘導する射は同型である.


\end{proof}



\begin{lemma}\label{pullback}
	Lem.\ref{effective_divisor}の状況で$F \in \Coh(X)$とすると$D^b(X)$におけるdistinguished triangle
	$$(F\otimes L_{|X})[1] \to Li^*(i_*F) \to F \xrightarrow{+1}$$がある.さらにこの図式の$Li^*(i_*F) \to F$は,随伴$Li^* \dashv i_*$に付随するcounit射$\varepsilon \colon Li^*(i_*F) \to F$である.
\end{lemma}
\begin{proof}
	$i_* \colon D^*(X) \to D^*(Y)$はexactだから$i_*Li^* \cong L(i_*i^*)$である.ここで$i_*i^* \cong -\otimes_{\mathcal{O}_Y}\mathcal{O}_X$だから,$L(i_*i^*) \cong -\otimes^L_{\mathcal{O}_Y}\mathcal{O}_X$となる.よってLem.\ref{effective_divisor}の完全列により$\mathcal{O}_X$を分解することで$$L(i_*i^*)(i_*F) \cong (\cdots \to 0 \to i_*F\otimes L \xrightarrow{d^{-1}} i_*F \to 0 \to \cdots)$$となる($i_*F$が$0$次のcomplex).このとき,$d^{-1}$が$0$射であることをしめす.問題はlocalなので$X, Y$はaffineとしてよい.$Y=\Spec{A}, X = \Spec{A/I}$とし,$F$は$A/I$加群$M$に付随する層だとする.Lem.\ref{effective_divisor}の完全列は$$0 \to A \to A \to A/I \to 0$$となり,左の射は$I$の生成元$f$による$f$倍写像である.すると$d^{-1}$は$f$倍写像$M \to M$となるが,$M$は$A/I$加群なのでこれは$0$射に等しい.

	ここまでで$$i_*Li^*(i_*F) \cong i_*F[0] \oplus i_*(F\otimes L)[1]$$が示せた.さらに$i_*$は完全関手だからコホモロジーをとる操作と可換であることに注意すると,複体$Li^*(i_*F)$の$j$次コホモロジー$\mathcal{H}^j$について$$i_*\mathcal{H}^0 \cong i_*F$$$$i_*\mathcal{H}^{-1}\cong i_*(F \otimes L_{|X})$$$$i_*\mathcal{H}^j = 0 \hspace{10pt}(j \neq 0, -1)$$となる.よって$$\mathcal{H}^0 \cong F$$$$\mathcal{H}^{-1}\cong F \otimes L_{|X}$$$$\mathcal{H}^j = 0 \hspace{10pt}(j \neq 0, -1)$$となる.ここでcounit射$\varepsilon \colon Li^*(i_*F) \to F$を補完するようなdistinguished triangle$$C\to Li^*(i_*F) \xrightarrow{\varepsilon} F \xrightarrow{+1}$$をとる.これに付随するコホモロジー長完全列をとると

	\[
		\begin{array}{ccccccc}
			 &     &                     &     & \cdots           & \to                                      & 0 \\
			 & \to & \mathcal{H}^{-2}(C) & \to & 0                & \to                                      & 0 \\
			 & \to & \mathcal{H}^{-1}(C) & \to & F \otimes L_{|X} & \to                                      & 0 \\
			 & \to & \mathcal{H}^{0}(C)  & \to & F                & \xrightarrow{\mathcal{H}^0(\varepsilon)} & F \\
			 & \to & \mathcal{H}^{1}(C)  & \to & 0                & \to                                      & 0 \\
			 & \to & \cdots              &     &                  &                                          &   \\
		\end{array}
	\]
	となり,Lem.\ref{counit_map}より$\mathcal{H}^0(\varepsilon)$は同型だから,$C \cong (F \otimes L_{|X})[1]$となる.
\end{proof}
\begin{lemma}\label{boxtimes_is_a_sheaf}
	$G \subset S$が$(-2)$-curveで$a \in \mathbb{Z}$のとき,$\mathcal{O}_G(a) \boxtimes \mathcal{O}_G(a)^\vee \in D^b(S\times S)$は$G\times G$上の層の$-1$シフト(のpushforward)である.
\end{lemma}
\begin{proof}
	$S$上のdivisor $D$であって$G.D = a$であるものを1つとる.($G.G = -2 \neq 0$とPoincaré dualityにより必ずとれる.)このとき$\mathcal{O}_S(D)_{|G} \cong \mathcal{O}_G(a)$である.完全列$$0 \to \mathcal{O}_S(-G) \to \mathcal{O}_S \to \mathcal{O}_G \to 0$$より,$D^b(S)$において$$\mathcal{O}_G(a) \cong (\cdots \to 0 \to \mathcal{O}_S(D-G) \to \mathcal{O}_S(D) \to 0 \to \cdots)$$となる(右辺は$0$次に$\mathcal{O}_S(D)$があるcomplex).よって
	\begin{align*}
		\mathcal{O}_G(a)^\vee & \cong (\cdots \to 0 \to \mathcal{O}_S(-D) \to \mathcal{O}_S(G-D) \to 0 \to \cdots) \\
		                      & \cong \mathcal{O}_S(G-D)_{|G}[-1]
	\end{align*}
	となる.すると
	\begin{align*}
		 & \mathcal{O}_G(a) \boxtimes \mathcal{O}_G(a)^\vee                                                                               \\
		 & \cong p_1^*\mathcal{O}_S(D)_{|G} \otimes^L p_2^*\mathcal{O}_S(G-D)_{|G}[-1]                                                    \\
		 & \cong \mathcal{O}_{S\times S}(D\times S)_{|G\times S} \otimes^L \mathcal{O}_{S\times S}(S\times G-S \times D)_{|S\times G}[-1] \\
	\end{align*}
	となる.よってderived tensorのhigher cohomologyが消えていることを示せばこれは
	\begin{align*}
		 & \mathcal{O}_{S\times S}(D\times S)_{|G\times S} \otimes \mathcal{O}_{S\times S}(S\times G-S \times D)_{|S\times G}[-1] \\
		 & = \mathcal{O}_{S\times S}(D\times S +S\times G-S \times D)_{|G\times G}[-1]
	\end{align*}
	となり命題が示される.問題はlocalなので$\mathcal{O}_{S\times S}(D\times S)_{|G\times S}$と$\mathcal{O}_{S\times S}(S\times G-S \times D)_{|S\times G}$はそれぞれ$\mathcal{O}_{G\times S}$と$\mathcal{O}_{S\times G}$だと思ってよく,すると$G \times S$と$S \times G$が$S \times S$の中でtransversal intersectionなので全ての$q>0$について$$\Torsheaf_q^{\mathcal{O}_{S\times S}}(\mathcal{O}_{G\times S}, \mathcal{O}_{S\times G})=0$$となりderived tensorのhigher cohomologyが消えていることがわかる.
\end{proof}
\begin{theorem}
	$\pi \colon S \to C$をelliptic surfaceとし,$G \subset S$を$(-2)$-curve,$a \in \mathbb{Z}$を整数とする.このとき$S$のspherical object $\mathcal{O}_G(a)$に付随するtwist functorの核$$P = \Cone(\mathcal{O}_G(a) \boxtimes \mathcal{O}_G(a)^\vee \xrightarrow{ev} \mathcal{O}_\Delta)$$は$S\times_C S$上のcomplex(より強く,層)のpushforwardである.
\end{theorem}
\begin{proof}
	Lem.\ref{boxtimes_is_a_sheaf}より,$G\times G$上の層$F$を用いて$\mathcal{O}_G(a) \boxtimes \mathcal{O}_G(a)^\vee \cong F[-1]$と表せる.よって$D^b(S \times S)$におけるdistinguished triangle$$F[-1] \xrightarrow{ev} \mathcal{O}_\Delta \to P \xrightarrow{+1} F$$がある.ここでもし$ev \in \Hom_{D^b(S\times S)}(F[-1], \mathcal{O}_\Delta)$が$\Hom_{D^b(S\times_C S)}(F[-1], \mathcal{O}_\Delta)$の元の像だったとすると,$D^b(S\times_C S)$でのCone$$P' = \Cone(\mathcal{O}_G(a) \boxtimes \mathcal{O}_G(a)^\vee \xrightarrow{ev} \mathcal{O}_\Delta)$$を$D^b(S\times S)$にpushしたものは$P$と同型になる.よってinclusion $S\times_C S \to S\times S$による(derived)pushforwardが誘導する射$$\Hom_{D^b(S\times_C S)}(F[-1], \mathcal{O}_\Delta) \to \Hom_{D^b(S\times S)}(F[-1], \mathcal{O}_\Delta)$$が全射であることを証明すれば定理が示される.

	以下$X=S\times_C S$,$ Y=S\times S$とおき,$i \colon X \to Y$を自然なinclusionとする.
	Lem.\ref{pullback}の distinguished triangle に $R\Hom(-, \mathcal{O}_\Delta)$をあてると
	$$R\Hom(F, \mathcal{O}_\Delta) \to R\Hom(Li^*(i_*F), \mathcal{O}_\Delta) \to R\Hom((F\otimes L_{|X})[1], \mathcal{O}_\Delta) \xrightarrow{+1}$$
	となり,コホモロジー長完全列をとることで
	\[
		\begin{array}{ccccccc}
			 &     &                               &     &                                        &     & 0                                         \\
			 & \to & \Ext^1(F, \mathcal{O}_\Delta) & \to & \Ext^1(Li^*(i_*F), \mathcal{O}_\Delta) & \to & \Hom(F\otimes L_{|X}, \mathcal{O}_\Delta) \\
			 & \to & \Ext^2(F, \mathcal{O}_\Delta) & \to & \cdots                                 &     &
		\end{array}
	\]
	という完全列を得る.この図式における$$\Ext^1(F, \mathcal{O}_\Delta) \to \Ext^1(Li^*(i_*F), \mathcal{O}_\Delta)$$はLem.\ref{pullback}より随伴$Li^* \dashv i_*$に付随するcounit射$\varepsilon \colon Li^*(i_*F) \to F$から誘導されるものだから,随伴同型$\Ext_Y^1(i_*F, i_*\mathcal{O}_\Delta) \cong \Ext_X^1(Li^*(i_*F), \mathcal{O}_\Delta)$によりpushforwardが誘導する射$$\Ext_X^1(F, \mathcal{O}_\Delta) \to \Ext_Y^1(i_*F, i_*\mathcal{O}_\Delta)$$と同一視される.よって上の長完全列より,$\Hom(F\otimes L_{|X}, \mathcal{O}_\Delta) =0$であることを証明すれば命題が示されることになる.

	以下それを示す.随伴により$$\Hom_X(F\otimes L_{|X}, \mathcal{O}_\Delta)\cong\Hom_\Delta((F\otimes L_{|X})_{|\Delta}, \mathcal{O}_\Delta)$$だが,$F$は$G \times G \subset X$上の層だったため,$(F\otimes L_{|X})_{|\Delta}$は$(G \times G) \cap \Delta = \Delta_G$上の層$F'$である.よって$$\Hom_\Delta((F\otimes L_{|X})_{|\Delta}, \mathcal{O}_\Delta)=\Hom_\Delta(F', \mathcal{O}_\Delta)$$となり,$\Delta_G \subset \Delta$は$G \subset S$とみなせるため,結局$G$上の層$F'$について$\Hom_S(F', \mathcal{O}_S)=0$を証明すればよい.これはSerre dualityより明らか.
\end{proof}
\begin{thebibliography}{99}
	\bibitem[Ueh15]{Ueh15} H. Uehara, \textit{Autoequivalences of derived categories of elliptic surfaces with non-zero Kodaira dimension},arXiv e-prints (2021), arXiv:1501.06657v2.
\end{thebibliography}
\end{document}